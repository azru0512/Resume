%%%%%%%%%%%%%%%%%%%%%%%%%%%%%%%%%%%%%%%%%
% Medium Length Professional CV
% LaTeX Template
% Version 2.0 (8/5/13)
%
% This template has been downloaded from:
% http://www.LaTeXTemplates.com
%
% Original author:
% Trey Hunner (http://www.treyhunner.com/)
%
% Important note:
% This template requires the resume.cls file to be in the same directory as the
% .tex file. The resume.cls file provides the resume style used for structuring the
% document.
%
%%%%%%%%%%%%%%%%%%%%%%%%%%%%%%%%%%%%%%%%%

%----------------------------------------------------------------------------------------
%	PACKAGES AND OTHER DOCUMENT CONFIGURATIONS
%----------------------------------------------------------------------------------------

\documentclass{resume} % Use the custom resume.cls style

\usepackage[left=0.75in,top=0.6in,right=0.75in,bottom=0.6in]{geometry} % Document margins

\name{Wei-Ren Chen} % Your name
\address{chenwj.cs97g@g2.nctu.edu.tw} % Your email

\begin{document}

%----------------------------------------------------------------------------------------
%	EDUCATION SECTION
%----------------------------------------------------------------------------------------

\begin{rSection}{Education}

{\bf National Chiao Tung University, Taiwan (R.O.C.)} \hfill {\em June 2010} \\ 
M.S. in Computer Science \& Engineering \\
Teaching Assistant of Computer Science Computer Center \\

\end{rSection}

%----------------------------------------------------------------------------------------
%	WORK EXPERIENCE SECTION
%----------------------------------------------------------------------------------------

\begin{rSection}{Work Experience}

\begin{rSubsection}{SGS-TUV Saar}{February 2022 - Present}{Functional Safety Project Manager}{New Taipei City, Taiwan}
\item ISO 26262 auditor and assessor.
\item ISO 21434 auditor.
\item Provide ISO 26262/ISO 21434/ISO 21448/ASPICE services.
\item IEEE SA member.
\begin{itemize}
\item IEEE P2851 WG Secretary.
\item IEEE FSSC voting member.
\end{itemize}
\end{rSubsection}

\begin{rSubsection}{Horizon Robotics Inc.}{August 2018 - February 2022}{Senior Software Engineer}{Shanghai, China}
\item Develop AI Toolchain for our AI accelerator, including front-end parsing and AST construction, middle-end instruction scheduling and combination, back-end code generation, and runtime development. 
\item Establish organization-wide policy and procedure complying with ISO 26262. After that, as the functional safety manager of our AI Toolchain project.
\item Internal trainer of ISO 26262 and ISO 21448.
\item IEEE SA member.
\begin{itemize}
\item IEEE P2851 WG member and SG co-leader.
\item IEEE FSSC voting member.
\end{itemize}
\end{rSubsection}

%------------------------------------------------

\begin{rSubsection}{Huawei Technologies Co. Ltd}{December 2015 - August 2018}{Senior Software Engineer}{Zhejiang, China}
\item Develop Clang/LLVM compiler, GNU binutils and GNU libgcc for our DSP.
\item Develop DSL for our wireless team, for programming L1/L2 of 4G/5G. The DSL is data-flow programming paradigm aims to help programmer focus on the in/out of each operations, and let the compiler do the code generation and optimizations. I involves in the language design, and implementing compiler (Scala) from the front-end to the back-end.
\end{rSubsection}

%------------------------------------------------

\begin{rSubsection}{Generalplus Technology Inc.}{November 2013 - December 2015}{Software Engineer}{Hsinchu, Taiwan}
\item Develop DSL for our 4-bit RISC-like processor. Instead of programming in the assembly, a DSL is implemented to help our customer reducing the time to market. The compiler is not only a standalone tool, but also integrated in our IDE to help our customer catching syntax and semantic errors in the early stage of development.
\item Maintain 6502 C compiler and tool chain. The 6502 C compiler is derived from SDCC, integrated with our proprietary tool chain. My responsibilities also include helping our FAE programming the C code, and writing technical report to our customer (in English of course). 
\end{rSubsection}

%------------------------------------------------

\begin{rSubsection}{Academia Sinica}{October 2010 - August 2013}{Research Assistant}{Taipei, Taiwan}
\item Research in compiler and virtualization techniques (QEMU and LLVM). Focus on binary translation and hardware performance profiling feedback optimization.
\end{rSubsection}

\end{rSection}

%----------------------------------------------------------------------------------------
%	PROFESSIONAL LICENSE
%----------------------------------------------------------------------------------------
\begin{rSection}{Professional License}

\begin{itemize}
\item intacs™ Provisional Assessor AUTOMOTIVE SPICE® - valid until Sep. 2024.
\item SGS-TUV Saar Automotive Functional Safety Experts - valid until Jun. 2025.
\item SGS-TUV Saar Certified Automotive Cyber Security Professionals - valid until Nov. 2025.
\item SGS-TUV Saar SOTIF Professionals - valid until Nov. 2025.
\end{itemize}

\end{rSection}

%----------------------------------------------------------------------------------------
%	OPEN SOURCE EXPERIENCE SECTION
%----------------------------------------------------------------------------------------

\begin{rSection}{Open Source Experience}

\item 
\begin{itemize}
\item Cygwin hidapi port maintainer.
\item Help LLVM community releasing its first ARM binary!
\end{itemize}

\end{rSection}

%----------------------------------------------------------------------------------------
%	TECHNICAL STRENGTHS SECTION
%----------------------------------------------------------------------------------------

%\begin{rSection}{Technical Strengths}

%\begin{tabular}{ @{} %>{\bfseries}l @{\hspace{6ex}} l }
%Computer Languages & C/C++, Scala %\\
%Tools & SVN/Git, Vim, Lex \& Yacc
%\end{tabular}

%\end{rSection}

\end{document}
